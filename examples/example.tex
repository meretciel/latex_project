\documentclass[letterpaper, oneside]{book}
%\usepackage{lingmacros}
\usepackage{tree-dvips}
\usepackage{amsmath}
\usepackage{listings}
\usepackage{graphicx}
\usepackage{xcolor}
\usepackage{mdframed}
\usepackage{fancyhdr}
\usepackage[export]{adjustbox}
\usepackage[skip=10pt]{parskip}
\usepackage{amsfonts}
\usepackage{hyperref}
\usepackage{algorithm}
\usepackage{algpseudocode}
\pagestyle{plain}



% Configuration

% Set the root directory for images and graphcis.
\graphicspath{{./images/}}

% Configure listings backage for code
\lstset{
	%numbers=left,
	%numberstyle=\tiny,
	breaklines=true,
	%numbersep=5pt,
    frame = single,
	xleftmargin=.15in,
	xrightmargin=.05in}

%\setlength{\parindent}{0pt}

% Define New Commands

\newcommand*{\vecthree}[3]{
	\begin{bmatrix}
		#1 \\ #2 \\ #3
\end{bmatrix}}

% Define New Environments




%\dfrac{num}{den}

\title{My first LaTeX document}
\author{Toby}
\date{September 2023}


\begin{document}
	\maketitle{}
	\tableofcontents



	\chapter{Basic}

	\section{Paragraph}

	This line is supposed to be a very long text. Its purpose is to show how the paragraph works in Latex. As you can see this is a multi-line text.

	To start a new paragraph, we can do a blank line in the latex  file. Each new paragraph has a default indent.

	This line is to show the effect of the indent mentioned in the previous paragraph.

	Some of the \textbf{greatest}
	discoveries in \underline{science}
	were made by \textbf{\textit{accident}}. hello

	Some of the greatest \emph{discoveries} in science
	were made by accident.

	\textit{Some of the greatest \emph{discoveries}
		in science were made by accident.}

	\textbf{Some of the greatest \emph{discoveries}
		in science were made by accident.}

	\section{Listing}

	Example: Unordered List \\
	\begin{itemize}
		\item The individual entries are indicated with a black dot, a so-called bullet.
		\item The text in the entries may be of any length.
	\end{itemize}

	Example: Ordered List: \\

	\begin{enumerate}
		\item This is the first entry in our list.
		\item The list numbers increase with each entry we add.
	\end{enumerate}


	\section{Links}

	Example: hyper link:  \href{https://news.ycombinator.com/news}{This is the link to a blog}

\section{Insert Code Snippets}

\subsection{Insert Code}

\begin{lstlisting}
enum AnyThing {
    Color(&'static str),
    Reddit(i32),
}

let obj1 = AnyThing::Color("This is a red color");
let obj2 = AnyThing::Reddit(10);

\end{lstlisting}

\subsection{Insert Algorithm}

\begin{algorithm}
	\caption{An algorithm with caption}\label{alg:cap}

	\begin{algorithmic}[1]
		\Require $n \geq 0$
		\Ensure $y = x^n$

		\State $y \gets 1$
		\State $X \gets x$
		\State $N \gets n$
		\While{$N \neq 0$}
		\If{$N$ is even}
		\State $X \gets X \times X$
		\State $N \gets \frac{N}{2}$  \Comment{This is a comment}
		\ElsIf{$N$ is odd}
		\State $y \gets y \times X$
		\State $N \gets N - 1$
		\EndIf
		\EndWhile


	\end{algorithmic}
\end{algorithm}



	\chapter{Images}
	Example: Use scale parameter \\
	\includegraphics[scale=0.6]{One_Piece_Logo.png}

	Example: Use max width and linewidth from adjustbox package. \\
	\includegraphics[max width=\linewidth]{One_Piece_Logo.png}

	Example: Use max width and textwidth from adjustbox package. \\
	\includegraphics[max width=\textwidth]{One_Piece_Logo.png}

	Example: Use figure and reference.
	\begin{figure}[h]
		\centering
		\includegraphics[width=0.75\textwidth]{One_Piece_Logo.png}
		\caption{A nice plot.}
		\label{fig:mesh1}
	\end{figure}

	As you can see in figure \ref{fig:mesh1}, the function grows near the origin. This example is on page \pageref{fig:mesh1}.




	\chapter{Math}


Example: Inline Math formula:  \\

In physics, the mass-energy equivalence is stated 
by the equation $E=mc^2$, discovered in 1905 by Albert Einstein. \\


Example: Inline Math formula 2: \\

\begin{math}
	E=mc^2
\end{math} is typeset in a paragraph using inline math mode---as is $E=mc^2$, and so too is \(E=mc^2\).

Example: Math Block: \\

The mass-energy equivalence is described by the famous equation
\[ E=mc^2 \] discovered in 1905 by Albert Einstein. 

In natural units ($c = 1$), the formula expresses the identity
\begin{equation}
	E=m
\end{equation}


\section{Matrix}

\[
\begin{matrix}
	1 & 2 & 3\\
	a & b & c
\end{matrix}
\]

\[
\begin{pmatrix}
	1 & 2 & 3\\
	a & b & c
\end{pmatrix}
\]

\[
\begin{bmatrix}
	1 & 2 & 3\\
	a & b & c
\end{bmatrix}
\]

\[
\begin{Bmatrix}
	1 & 2 & 3\\
	a & b & c
\end{Bmatrix}
\]

\[
\begin{Bmatrix}
	1 & 2 & 3\\
	a & b & c
\end{Bmatrix}
\]

\[
\begin{Vmatrix}
	1 & 2 & 3\\
	a & b & c
\end{Vmatrix}
\]

\section{Equations}

\subsection{Single Line Equation}
\begin{equation}
	e^{\pi i} + 1 = 0
\end{equation}

\subsection{Multi-Line Equation}
\begin{equation}
\begin{split}
e^{\pi i} + 1 & = 0 \\
 				    & = 0 + 1 + 2 - 1 - 2 
\end{split}
\end{equation}

\subsection{Multi-line formula}
\begin{align*}
	e^{\pi i} + 1 & =  0 \\
	e^{\pi i} + 1 & =  0 
\end{align*}



	\chapter{Table}

\section{Basic Table}
\begin{center}
	\begin{tabular}{c c c}
		cell1 & cell2 & cell3 \\ 
		cell4 & cell5 & cell6 \\  
		cell7 & cell8 & cell9    
	\end{tabular}
\end{center}

\section{Table with Boarder}

\begin{tabular}{|c|c|c|} 
	\hline
	cell1 & cell2 & cell3 \\ 
	cell4 & cell5 & cell6 \\ 
	cell7 & cell8 & cell9 \\ 
	\hline
\end{tabular}

\section{Table with Caption}

Table \ref{table:data} shows how to add a table caption and reference a table.
\begin{table}[h!]
	\centering
	\begin{tabular}{||c c c c||} 
		\hline
		Col1 & Col2 & Col2 & Col3 \\ [0.5ex] 
		\hline\hline
		1 & 6 & 87837 & 787 \\ 
		2 & 7 & 78 & 5415 \\
		3 & 545 & 778 & 7507 \\
		4 & 545 & 18744 & 7560 \\
		5 & 88 & 788 & 6344 \\ [1ex] 
		\hline
	\end{tabular}
	\caption{Table to test captions and labels.}
	\label{table:data}
\end{table}


\section{My Default Table Style}

\begin{table}[h!]
	\centering
	\begin{tabular}{|c c c c|}
		\hline
		Item & Status & Description & Note \\
		\hline
		
		1 & OK & this is the first test & nan \\
		2 & Failed & The second test failed & Need to retry \\
		3 & Pending & Pending for testing & Pending \\
		
		\hline
	\end{tabular}
	\caption{test caption}
	\label{table:test-label}
\end{table}
	
\newcommand*{\R}{\mathbb{R}}
\newcommand*{\bb}[1]{\mathbb{#1}}
\newcommand{\plusbinomial}[3][2]{(#2 + #3)^#1}

\newenvironment{xboxed}[2][This is a box]
{\begin{center}
		Argument 1 (\#1)=#1\\[1ex]
		\begin{tabular}{{{!}}p{0.9\textwidth}{{!}}}
			\hline\\
			Argument 2 (\#2)=#2\\[2ex]
		}
		{ 
			\\\\\hline
		\end{tabular} 
	\end{center}
}

\chapter[Command and Environment]{Command and \rlap{Environment}}

The set of real numbers are usually represented 
by a blackboard bold capital R: \( \R \).

Other numerical systems have similar notations. 
The complex numbers \( \bb{C} \), the rational 
numbers \( \bb{Q} \) and the integer numbers \( \bb{Z} \).

We can use it like this: \[ \plusbinomial{x}{y} \]

And even the exponent can be changed:

\[ \plusbinomial[4]{a}{b} \]

	\chapter{Robotics}


\begin{algorithm}
    \renewcommand{\thealgorithm}{} % this line is used to remove the algorithm number.
    \caption{\textbf{Bayes Filter}}\label{alg:cap}

    \begin{algorithmic}[1]
        \Function{BayesFilter}{$bel(x_{t-1}), u_t, z_t$}
        \ForAll{$x_t$}
        \State $\overline{bel}(x_t) = \int{p(x_t|u_t, x_{t-1})bel(x_{t-1})dx_{t-1}}$
        \State $bel(x_t) = \eta{}p(z_t|x_t)\overline{bel}(x_t)$
        \EndFor
        \State \textbf{return} $bel(x_t)$
        \EndFunction

    \end{algorithmic}

\end{algorithm}

\newpage
\begin{algorithm}
    \renewcommand{\thealgorithm}{}
    \begin{algorithmic}[1]
    \Function{Kalman\_Filter}{$\mu_{t-1}, \Sigma_{t-1}, u_t, z_t$}
    \State $\bar{\mu}_t = A_t \mu_{t-1} + B_t u_t$
    \State $\bar{\Sigma}_t = A_t\Sigma_{t-1}A_t^T + R_t$

    \State $K_t = \bar{\Sigma}_t C_t^T (C_t \bar{\Sigma}_t  C_t^T + Q_t)^{-1}$  \Comment{This is called Kalman gain}

    \State $\mu_t = \bar{\mu}_t + K_t(z_t - C_t \bar{\mu}_t)$

    \State $\Sigma_t = (I - K_tC_t) \bar{\Sigma}_t$

    \State $\textbf{return} \;\; \mu_t, \Sigma_t$
    \EndFunction

\end{algorithmic}
\end{algorithm}



\begin{align*}
x_t & = g(u_t, x_{t-1}) + \epsilon_t \\
z_t & = h(x_t) + \delta_t
\end{align*}

\[
g(u_t, x_{t-1}) \approx g(u_t, \mu_{t-1}) + \frac{}{}(u_t, \mu_{t-1})(x_{t-1} - \mu_{t-1})
\]

\begin{equation}
    \begin{split}
    g(u_t, x_{t-1}) & \approx g(u_t, \mu_{t-1}) + \frac{\partial g}{\partial x_{t-1}}(u_t, \mu_{t-1})(x_{t-1} - \mu_{t-1}) \\
                    & = g(u_t, \mu_{t-1}) + G_t(x_{t-1} - \mu_{t-1})
    \end{split}
\end{equation}



\begin{align*}
    g(u_t, x_{t-1}) & \approx g(u_t, \mu_{t-1}) + \frac{\partial g}{\partial x_{t-1}}(u_t, \mu_{t-1})(x_{t-1} - \mu_{t-1}) \\
    & = g(u_t, \mu_{t-1}) + G_t(x_{t-1} - \mu_{t-1})
\end{align*}

\[
h(x_t) \approx h(\bar{\mu}_t) + H_t(x_t - \bar{\mu}_t)
\]


%\bar{\mu}_t = g(u_t, \mu_{t-1})



\begin{algorithm}
    \renewcommand{\thealgorithm}{}
    \begin{algorithmic}[1]
        \Function{EKF}{$\mu_{t-1}, \Sigma_{t-1}, u_t, z_t$}
        \State $\bar{\mu}_t = g(u_t, \mu_{t-1})$
        \State $\bar{\Sigma}_t = G_t\Sigma_{t-1}G_t^T + R_t$

        \State $K_t = \bar{\Sigma}_t H_t^T (H_t \bar{\Sigma}_t  H_t^T + Q_t)^{-1}$

        \State $\mu_t = \bar{\mu}_t + K_t(z_t - h(\bar{\mu}_t))$

        \State $\Sigma_t = (I - K_tH_t) \bar{\Sigma}_t$

        \State $\textbf{return} \;\; \mu_t, \Sigma_t$
        \EndFunction

    \end{algorithmic}
\end{algorithm}





	
	
\chapter{Examples}

\begin{math}
	f(x_1, x_2) = 100 (x_2 - x_{1}^2)^2 + (1 - x_1)^2
\end{math}


\end{document}

}

