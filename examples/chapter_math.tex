\chapter{Math}


Example: Inline Math formula:  \\

In physics, the mass-energy equivalence is stated 
by the equation $E=mc^2$, discovered in 1905 by Albert Einstein. \\


Example: Inline Math formula 2: \\

\begin{math}
	E=mc^2
\end{math} is typeset in a paragraph using inline math mode---as is $E=mc^2$, and so too is \(E=mc^2\).

Example: Math Block: \\

The mass-energy equivalence is described by the famous equation
\[ E=mc^2 \] discovered in 1905 by Albert Einstein. 

In natural units ($c = 1$), the formula expresses the identity
\begin{equation}
	E=m
\end{equation}


\section{Matrix}

\[
\begin{matrix}
	1 & 2 & 3\\
	a & b & c
\end{matrix}
\]

\[
\begin{pmatrix}
	1 & 2 & 3\\
	a & b & c
\end{pmatrix}
\]

\[
\begin{bmatrix}
	1 & 2 & 3\\
	a & b & c
\end{bmatrix}
\]

\[
\begin{Bmatrix}
	1 & 2 & 3\\
	a & b & c
\end{Bmatrix}
\]

\[
\begin{Bmatrix}
	1 & 2 & 3\\
	a & b & c
\end{Bmatrix}
\]

\[
\begin{Vmatrix}
	1 & 2 & 3\\
	a & b & c
\end{Vmatrix}
\]

\section{Equations}

\subsection{Single Line Equation}
\begin{equation}
	e^{\pi i} + 1 = 0
\end{equation}

\subsection{Multi-Line Equation}
\begin{equation}
\begin{split}
e^{\pi i} + 1 & = 0 \\
 				    & = 0 + 1 + 2 - 1 - 2 
\end{split}
\end{equation}

\subsection{Multi-line formula}
\begin{align*}
	e^{\pi i} + 1 & =  0 \\
	e^{\pi i} + 1 & =  0 
\end{align*}


